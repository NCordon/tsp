\documentclass[ignorenonframetext,]{beamer}
\setbeamertemplate{caption}[numbered]
\setbeamertemplate{caption label separator}{:}
\setbeamercolor{caption name}{fg=normal text.fg}
\usepackage{amssymb,amsmath}
\usepackage{ifxetex,ifluatex}
\usepackage{fixltx2e} % provides \textsubscript
\usepackage{lmodern}
\ifxetex
  \usepackage{fontspec,xltxtra,xunicode}
  \defaultfontfeatures{Mapping=tex-text,Scale=MatchLowercase}
  \newcommand{\euro}{€}
\else
  \ifluatex
    \usepackage{fontspec}
    \defaultfontfeatures{Mapping=tex-text,Scale=MatchLowercase}
    \newcommand{\euro}{€}
  \else
    \usepackage[T1]{fontenc}
    \usepackage[utf8]{inputenc}
      \fi
\fi
\geometry{paperwidth=200mm,paperheight=160mm}
\usetheme{Berlin}
\usecolortheme{beaver}
% use upquote if available, for straight quotes in verbatim environments
\IfFileExists{upquote.sty}{\usepackage{upquote}}{}
% use microtype if available
\IfFileExists{microtype.sty}{\usepackage{microtype}}{}
\usepackage{color}
\usepackage{fancyvrb}
\newcommand{\VerbBar}{|}
\newcommand{\VERB}{\Verb[commandchars=\\\{\}]}
\DefineVerbatimEnvironment{Highlighting}{Verbatim}{commandchars=\\\{\}}
% Add ',fontsize=\small' for more characters per line
\newenvironment{Shaded}{}{}
\newcommand{\KeywordTok}[1]{\textcolor[rgb]{0.00,0.44,0.13}{\textbf{{#1}}}}
\newcommand{\DataTypeTok}[1]{\textcolor[rgb]{0.56,0.13,0.00}{{#1}}}
\newcommand{\DecValTok}[1]{\textcolor[rgb]{0.25,0.63,0.44}{{#1}}}
\newcommand{\BaseNTok}[1]{\textcolor[rgb]{0.25,0.63,0.44}{{#1}}}
\newcommand{\FloatTok}[1]{\textcolor[rgb]{0.25,0.63,0.44}{{#1}}}
\newcommand{\CharTok}[1]{\textcolor[rgb]{0.25,0.44,0.63}{{#1}}}
\newcommand{\StringTok}[1]{\textcolor[rgb]{0.25,0.44,0.63}{{#1}}}
\newcommand{\CommentTok}[1]{\textcolor[rgb]{0.38,0.63,0.69}{\textit{{#1}}}}
\newcommand{\OtherTok}[1]{\textcolor[rgb]{0.00,0.44,0.13}{{#1}}}
\newcommand{\AlertTok}[1]{\textcolor[rgb]{1.00,0.00,0.00}{\textbf{{#1}}}}
\newcommand{\FunctionTok}[1]{\textcolor[rgb]{0.02,0.16,0.49}{{#1}}}
\newcommand{\RegionMarkerTok}[1]{{#1}}
\newcommand{\ErrorTok}[1]{\textcolor[rgb]{1.00,0.00,0.00}{\textbf{{#1}}}}
\newcommand{\NormalTok}[1]{{#1}}

% Comment these out if you don't want a slide with just the
% part/section/subsection/subsubsection title:
\AtBeginPart{
  \let\insertpartnumber\relax
  \let\partname\relax
  \frame{\partpage}
}
\AtBeginSection{
  \let\insertsectionnumber\relax
  \let\sectionname\relax
  \frame{\sectionpage}
}
\AtBeginSubsection{
  \let\insertsubsectionnumber\relax
  \let\subsectionname\relax
  \frame{\subsectionpage}
}

\setlength{\parindent}{0pt}
\setlength{\parskip}{6pt plus 2pt minus 1pt}
\setlength{\emergencystretch}{3em}  % prevent overfull lines
\setcounter{secnumdepth}{0}


\begin{document}

\section{Travelling Salesman Problem}\label{travelling-salesman-problem}

\textbf{Ignacio Cordón Castillo}, \emph{Doble Grado Matemáticas e
Ingeniería Informática, Metaheurísticas, UGR}

\begin{frame}[fragile]{Representación de soluciones}

Definimos una solución como un objeto \texttt{Route}, que almacena una
permutación de ciudades, una matriz de distancia, y contiene métodos de
cálculo del coste y de 2-opt de arcos y ciudades

\begin{Shaded}
\begin{Highlighting}[]
\KeywordTok{class} \NormalTok{Route:  }
    \KeywordTok{def} \OtherTok{__init__}\NormalTok{(}\OtherTok{self}\NormalTok{, permutation, dist):}
        \OtherTok{self}\NormalTok{.permutation = deepcopy(array(permutation))}
        \OtherTok{self}\NormalTok{.dist = array(dist)}
        \OtherTok{self}\NormalTok{.update_cost()}

    \KeywordTok{def} \NormalTok{update_cost(}\OtherTok{self}\NormalTok{):}
        \NormalTok{pairs = }\OtherTok{self}\NormalTok{.get_edges()}
        \OtherTok{self}\NormalTok{.cost = }\DataTypeTok{sum}\NormalTok{([}\OtherTok{self}\NormalTok{.dist[x,y] }\KeywordTok{for} \NormalTok{(x,y) in pairs])}

    \KeywordTok{def} \NormalTok{change_edges(}\OtherTok{self}\NormalTok{,i,j)}
        \NormalTok{i,j = }\DataTypeTok{min}\NormalTok{(i,j), }\DataTypeTok{max}\NormalTok{(i,j)}

        \NormalTok{rev = }\OtherTok{self}\NormalTok{.permutation[i:j]}
        \NormalTok{rev = rev[::-}\DecValTok{1}\NormalTok{]}
        \OtherTok{self}\NormalTok{.permutation[i:j] = rev}
        \OtherTok{self}\NormalTok{.update_cost()}

    \KeywordTok{def} \NormalTok{get_edges(}\OtherTok{self}\NormalTok{):}
        \NormalTok{shifted = append(}\OtherTok{self}\NormalTok{.permutation[}\DecValTok{1}\NormalTok{:], [}\OtherTok{self}\NormalTok{.permutation[}\DecValTok{0}\NormalTok{]])}
        \NormalTok{pairs = }\DataTypeTok{zip}\NormalTok{(}\OtherTok{self}\NormalTok{.permutation, shifted)}
        \KeywordTok{return}\NormalTok{(pairs)}
            
\end{Highlighting}
\end{Shaded}

\end{frame}

\begin{frame}{Enfriamiento simulado}

Emplea esquema de enfriamiento geométrico, con operador de generación de
vecino 2-opt de arcos, y selección de temperatura inicial a:

\[T_0 = \frac{\mu}{-log(\phi)} \cdot Cost(S_0)  \]

\begin{itemize}
\item
  \textbf{\texttt{n\_iter}}: Número de evaluaciones que se hacen de la
  función objetivo
\item
  \textbf{\texttt{max\_exitos}}: Número de máximo de mejoras encontradas
  en un vecindario
\item
  \textbf{\texttt{max\_vecinos}}: Número de vecinos que se generarán en
  cada exploración de vecindario
\item
  \textbf{\texttt{alpha}}: factor de descenso de la media geométrica
  $T_k = \alpha \cdot T_{k-1}$
\item
  \textbf{\texttt{mu}}
\item
  \textbf{\texttt{phi}}
\end{itemize}

\end{frame}

\begin{frame}[fragile]{Enfriamiento simulado}

\begin{Shaded}
\begin{Highlighting}[]
\KeywordTok{def} \NormalTok{simulated_annealing(}\OtherTok{self}\NormalTok{, max_iter, max_exitos, max_vecinos, alpha, mu, phi):}
        
    \KeywordTok{while} \NormalTok{num_iter < max_iter:           }
        \KeywordTok{while} \NormalTok{num_iter < max_iter and num_vecinos < max_vecinos and exitos < max_exitos:}

            \NormalTok{candidate = deepcopy(solution)}
            \NormalTok{u = randint(}\DecValTok{0}\NormalTok{, n)}
            \NormalTok{v = randint(}\DecValTok{0}\NormalTok{, n)   }
            \NormalTok{candidate.change_edges(u,v)}

            \NormalTok{diff_cost = candidate.cost - solution.cost}

            \KeywordTok{if} \NormalTok{(diff_cost < }\DecValTok{0} \NormalTok{or random() < exp(-diff_cost*}\FloatTok{1.0}\NormalTok{/t)):}
                \NormalTok{solution = deepcopy(candidate)}
                \NormalTok{exitos+=}\DecValTok{1}

                \KeywordTok{if} \NormalTok{(solution.cost < best_solution.cost):}
                    \NormalTok{best_solution = deepcopy(solution)}

            \NormalTok{num_iter+=}\DecValTok{1}
            \NormalTok{num_vecinos+=}\DecValTok{1}

        \NormalTok{t = alpha*t}

    \KeywordTok{return} \NormalTok{best_solution}
    
\end{Highlighting}
\end{Shaded}

\end{frame}

\end{document}
